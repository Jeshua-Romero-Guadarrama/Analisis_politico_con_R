% Options for packages loaded elsewhere
\PassOptionsToPackage{unicode}{hyperref}
\PassOptionsToPackage{hyphens}{url}
%
\documentclass[
]{book}
\usepackage{amsmath,amssymb}
\usepackage{lmodern}
\usepackage{ifxetex,ifluatex}
\ifnum 0\ifxetex 1\fi\ifluatex 1\fi=0 % if pdftex
  \usepackage[T1]{fontenc}
  \usepackage[utf8]{inputenc}
  \usepackage{textcomp} % provide euro and other symbols
\else % if luatex or xetex
  \usepackage{unicode-math}
  \defaultfontfeatures{Scale=MatchLowercase}
  \defaultfontfeatures[\rmfamily]{Ligatures=TeX,Scale=1}
\fi
% Use upquote if available, for straight quotes in verbatim environments
\IfFileExists{upquote.sty}{\usepackage{upquote}}{}
\IfFileExists{microtype.sty}{% use microtype if available
  \usepackage[]{microtype}
  \UseMicrotypeSet[protrusion]{basicmath} % disable protrusion for tt fonts
}{}
\makeatletter
\@ifundefined{KOMAClassName}{% if non-KOMA class
  \IfFileExists{parskip.sty}{%
    \usepackage{parskip}
  }{% else
    \setlength{\parindent}{0pt}
    \setlength{\parskip}{6pt plus 2pt minus 1pt}}
}{% if KOMA class
  \KOMAoptions{parskip=half}}
\makeatother
\usepackage{xcolor}
\IfFileExists{xurl.sty}{\usepackage{xurl}}{} % add URL line breaks if available
\IfFileExists{bookmark.sty}{\usepackage{bookmark}}{\usepackage{hyperref}}
\hypersetup{
  pdftitle={Analisis político con R},
  pdfauthor={Jeshua Romero Guadarrama},
  hidelinks,
  pdfcreator={LaTeX via pandoc}}
\urlstyle{same} % disable monospaced font for URLs
\usepackage{longtable,booktabs,array}
\usepackage{calc} % for calculating minipage widths
% Correct order of tables after \paragraph or \subparagraph
\usepackage{etoolbox}
\makeatletter
\patchcmd\longtable{\par}{\if@noskipsec\mbox{}\fi\par}{}{}
\makeatother
% Allow footnotes in longtable head/foot
\IfFileExists{footnotehyper.sty}{\usepackage{footnotehyper}}{\usepackage{footnote}}
\makesavenoteenv{longtable}
\usepackage{graphicx}
\makeatletter
\def\maxwidth{\ifdim\Gin@nat@width>\linewidth\linewidth\else\Gin@nat@width\fi}
\def\maxheight{\ifdim\Gin@nat@height>\textheight\textheight\else\Gin@nat@height\fi}
\makeatother
% Scale images if necessary, so that they will not overflow the page
% margins by default, and it is still possible to overwrite the defaults
% using explicit options in \includegraphics[width, height, ...]{}
\setkeys{Gin}{width=\maxwidth,height=\maxheight,keepaspectratio}
% Set default figure placement to htbp
\makeatletter
\def\fps@figure{htbp}
\makeatother
\setlength{\emergencystretch}{3em} % prevent overfull lines
\providecommand{\tightlist}{%
  \setlength{\itemsep}{0pt}\setlength{\parskip}{0pt}}
\setcounter{secnumdepth}{5}
\usepackage{amsthm}
\usepackage{float}
\usepackage{rotating, graphicx}
\usepackage{multirow}
\usepackage{tabularx}

% new command for pretty oversets with \sim
\newcommand\simcal[1]{\stackrel{\sim}{\smash{\mathcal{#1}}\rule{0pt}{0.5ex}}}

\newcommand{\comma}{,\,}

\floatplacement{figure}{H}

\PassOptionsToPackage{table}{xcolor}

\usepackage{tcolorbox}

\definecolor{kcblue}{HTML}{D7DDEF}
\definecolor{kcdarkblue}{HTML}{2B4E70}

\makeatletter
\def\thm@space@setup{%
  \thm@preskip=8pt plus 2pt minus 4pt
  \thm@postskip=\thm@preskip
}
\makeatother

% \makeatletter % undo the wrong changes made by mathspec
% \let\RequirePackage\original@RequirePackage
% \let\usepackage\RequirePackage
% \makeatother

\newenvironment{rmdknit}
    {\begin{center}
    \begin{tabular}{|p{0.9\textwidth}|}
    \hline\\
    }
    {
    \\\\\hline
    \end{tabular}
    \end{center}
    }

\newenvironment{rmdnote}
    {\begin{center}
    \begin{tabular}{|p{0.9\textwidth}|}
    \hline\\
    }
    {
    \\\\\hline
    \end{tabular}
    \end{center}
    }

\newtcolorbox[auto counter, number within=section]{keyconcepts}[2][]{%
colback=kcblue,colframe=kcdarkblue,fonttitle=\bfseries, title=Key Concept~#2, after title={\newline #1}, beforeafter skip=15pt}
\ifluatex
  \usepackage{selnolig}  % disable illegal ligatures
\fi
\usepackage[]{natbib}
\bibliographystyle{apalike}

\title{Analisis político con R}
\author{Jeshua Romero Guadarrama}
\date{2021-07-30}

\begin{document}
\maketitle

{
\setcounter{tocdepth}{1}
\tableofcontents
}
\hypertarget{prefacio}{%
\chapter*{Prefacio}\label{prefacio}}
\addcontentsline{toc}{chapter}{Prefacio}

Publicado por Jeshua Romero Guadarrama en colaboración con JeshuaNomics:

{ Git Hub}
{ Facebook}
{ Twitter}
{ Linkedin}
{ Vkontakte}
{ Tumblr}
{ YouTube}
{ Instagram}

Jeshua Romero Guadarrama es economista y actuario por la Universidad Nacional Autónoma de México, quien ha construido el presente proyecto en colaboración con JeshuaNomics, ubicado en la Ciudad de México, se puede contactar mediante el siguiente correo electrónico: \href{mailto:jeshuanomics@gmail.com}{\nolinkurl{jeshuanomics@gmail.com}}.
Última actualización el viernes 30 del 07 de 2021

Los estudiantes con poca experiencia en el análisis avanzado de políticas a menudo tienen dificultades para entender los beneficios de desarrollar habilidades de programación estadística en \textbf{R} al momento de aplicar diversos métodos descriptivos e inferenciales. Análisis político con R por Jeshua Romero Guadarrama (2021), ofrece una introducción interactiva a los aspectos esenciales de la programación por medio del lenguaje y software estadístico \textbf{R}, así como una guía para la aplicación de la teoría política y análisis detallado de políticas públicas en entornos específicos. En otras palabras, el objetivo es que los estudiantes se adentren al mundo de la política aplicada mediante ejemplos empíricos presentados en la vida diaria y haciendo uso de las habilidades de programación recién adquiridas. Dicho objetivo se encuentra respaldado por ejercicios de programación interactivos y la incorporación de visualizaciones dinámicas de conceptos fundamentales mediante la flexibilidad de JavaScript, a través de la biblioteca D3.js.

En los últimos años, el lenguaje de programación estadística \textbf{R} se ha convertido en una parte integral del plan de estudios de las clases de análisis político y estadística que se imparten en las universidades. Regularmente una gran parte de los estudiantes no han estado expuestos a ningún lenguaje de programación antes y, por lo tanto, tienen dificultades para participar en el aprendizaje de \textbf{R} por sí mismos. Con poca experiencia en el análisis avanzado de estadísticas, es natural que los novicios tengan dificultades para comprender los beneficios de desarrollar habilidades en \textbf{R} para aprender y aplicar la estadística. Estos incluyen particularmente la capacidad de realizar, documentar y comunicar estudios empíricos y tener las facilidades para programar estudios de simulación, lo cual es útil para, por ejemplo, comprender y validar teoremas que generalmente no se asimilan o entienden fácilmente con el estudio de las fórmulas. Al ser un economista aplicado y analista político, me gustaría que mis colegas desarrollen capacidades de gran valor; en consecuencia, deseo compartir con las nuevas generaciones de politólogos y economistas mis conocimientos.

En lugar de confrontar a los estudiantes con ejercicios de codificación puros y literatura clásica complementaria, he pensado que sería mejor proporcionar material de aprendizaje interactivo que combine el código en \textbf{R} con el contenido del curso de texto \emph{Introducción a la Econometría} de \citet{stock2015} que sirve de base para el presente material. El presente trabajo es un complemento empírico interactivo al estilo de un informe de investigación reproducible que permite a los estudiantes no solo aprender cómo los resultados de los estudios de casos se pueden replicar con \textbf{R}, sino que también fortalece su capacidad para utilizar las habilidades recién adquiridas en otras aplicaciones empíricas.

\hypertarget{las-convenciones-usadas-en-el-presente-curso}{%
\subsubsection*{Las convenciones usadas en el presente curso}\label{las-convenciones-usadas-en-el-presente-curso}}
\addcontentsline{toc}{subsubsection}{Las convenciones usadas en el presente curso}

\begin{itemize}
\item
  El texto \emph{en cursiva} indica nuevos términos, nombres, botones y similares.
\item
  El texto \textbf{en negrita} se usa generalmente en párrafos para referirse al código \textbf{R}. Esto incluye comandos, variables, funciones, tipos de datos, bases de datos y nombres de archivos.
\item
  Texto de ancho constante sobre fondo gris indica un código \textbf{R} que usted puede escribir literalmente. Puede aparecer en párrafos para una mejor distinción entre declaraciones de código ejecutables y no ejecutables, pero se encontrará principalmente en forma de grandes bloques de código \textbf{R}. Estos bloques se denominan fragmentos de código.
\end{itemize}

\hypertarget{reconocimiento}{%
\subsubsection*{Reconocimiento}\label{reconocimiento}}
\addcontentsline{toc}{subsubsection}{Reconocimiento}

A mi alma máter: Universidad Nacional Autónoma de México (Facultad de Economía y Facultad de Ciencias). Por brindarme valiosas oportunidades que coadyuvaron a mi formación.

\begin{itemize}
\tightlist
\item
  ADORNO, Theodor W. ``LA PERSONALIDAD AUTORITARIA'', Paidós, Bs.As., 1969.
\item
  ALIGHIERI, Dante ``DE LA MONARQUIA'', Losada, Bs.As.
\item
  ALBERIONI et al.~``L'ATTIVISTA DI PARTITO'', Bolonia, 1967.
\item
  ALMOND, G. y COLEMAN ``THE POLITICS OF THE DEVELOPING AREAS'', Princeton
  University Press, 1960.
\item
  ALMOND, G.A. y VERBA, Sidney ``THE CIVIC CULTURE'', Princeton University Press, 1963.
\item
  ALMOND, G.A. y POWELL, G.B. ``POLITICA COMPARADA'', Paidós, Bs.As., 1972.
\item
  ALTHUSSER, Louis ``PARA LEER EL CAPITAL'', Siglo XXI, México, 1969.
\item
  AMIN, Samir ``SOBRE EL DESARROLLO DESIGUAL DE LAS FORMACIONES SOCIALES'',
  Anagrama, Barcelona, 1976.
\item
  AMIN, Samir ``L'ACCUMULATION A L'ECHELLE MONDIALE'', Editions Anthropos, París,
\end{itemize}

\begin{enumerate}
\def\labelenumi{\arabic{enumi}.}
\setcounter{enumi}{1969}
\tightlist
\item
\end{enumerate}

\begin{itemize}
\tightlist
\item
  ANDERSON, Perry ``CONSIDERACIONES SOBRE EL MARXISMO OCCIDENTAL'', Siglo XXI,
  México, 1990.
\item
  APONTE, Antonio ``LA ECONOMIA DE LOS PAISES SOCIALISTAS'', Salvat ed., Barcelona,
\end{itemize}

\begin{enumerate}
\def\labelenumi{\arabic{enumi}.}
\setcounter{enumi}{1972}
\tightlist
\item
\end{enumerate}

\begin{itemize}
\tightlist
\item
  ARENDT, Hannah ``LOS ORIGENES DEL TOTALITARISMO'', Alianza ed., Madrid, 1981-1982,
  2 vol.
\item
  APTER, David E. ``POLITICA DE LA MODERNIZACION'', Paidós, Bs.As., 1972.
\item
  ARISTOTELES ``LA POLITICA'', Editora Nacional, Madrid, 1977.
\item
  ARNOLETTO, Eduardo J. ``APROXIMACION A LA CIENCIA POLITICA'', Artesol ed., Córdoba,
\end{itemize}

\begin{enumerate}
\def\labelenumi{\arabic{enumi}.}
\setcounter{enumi}{1988}
\tightlist
\item
\end{enumerate}

\begin{itemize}
\tightlist
\item
  ARON, Raymond ``EL OPIO DE LOS INTELECTUALES'', Siglo XX, Bs.As., 1968.
\item
  ARON, Raymond ``DEMOCRACIA Y TOTALITARISMO'', Seix, Barcelona, 1971.
\item
  ARON, Raymond ``PAZ Y GUERRA ENTRE LAS NACIONES'', Alianza ed., Madrid, 1984.
\item
  ARON, Raymond ``REPUBLIQUE IMPERIALE: LES ETATS-UNIS DANS LE MONDE (1945-
  1972)'', Calmann-Lévy, París, 1973.
\item
  ``BREVE DICCIONARIO POLITICO'', Ed. Progreso, Moscú, 1983.
\item
  BALANDIER, George ``ANTROPOLOGIA POLITICA'', Península, Barcelona, 1969.- BARAN, Paul y SWEEZY, Paul ``EL CAPITALISMO MONOPOLISTA. UN ENSAYO SOBRE LA
  SOCIEDAD INDUSTRIAL AMERICANA'', Siglo XXI, México, 1968.
\item
  BARRACLOUGH, Geofrey ``UNE INTRODUCTION A L'HISTOIRE CONTEMPORAINE'', Ed.
  Stock, París, 1964.
\item
  BELL, Daniel ``FIN DE LAS IDEOLOGIAS'', Tecnos, Madrid, 1964.
\item
  BELL, Daniel ``THE END OF IDEOLOGY: ON THE EXHAUSTION OF POLITICAL IDEAS IN
  THE FIFTIES'', New York, 1960.
\item
  BENDIX, R. ``NATION-BUILDING AND CITIZENSHIP'', John Wiley, New York, 1964.
\item
  BERTALANFFY, Ludwig von ``TEORIA GENERAL DE LOS SISTEMAS'', FCE, México, 1981.
\item
  BENOIST, Alain de ``DEMOCRATIE: LE PROBLEME'', Le Labyrinthe, París, 1985.
\item
  BEYME, Klaus von ``TEORIAS POLITICAS CONTEMPORANEAS - UNA INTRODUCCION'',
  Instituto de Estudios Políticos, Madrid, 1977.
\item
  BOBBIO, N. et al.~``DICCIONARIO DE POLITICA'', Siglo XXI, México, 1986.
\item
  BOBBIO, Norberto ``SAGGI SULLA SCIENZA POLITICA IN ITALIA'', Bari, 1969.
\item
  BOBBIO, Norberto ``IL FUTURO DELLA DEMOCRAZIA. UNA DIFESA DELLE REGOLE DEL
  GIOCO'', Einaudi ed., Torino, 1984.
\item
  LE BON, Gustave ``PSICOLOGIA DE LAS MULTITUDES'', Ed. Albatros, Bs.As., 1978.
\item
  BRZEZINSKI, Zbigniew ``IDEOLOGIA Y PODER EN LA POLITICA SOVIETICA'', Paidós,
  Bs.As., 1968.
\item
  BRAILLARD, Philippe ``THEORIE DES SYSTEMES ET RELATIONS INTERNA-TIONALES'',
  Ed. Bruylant, Bruselas, 1977.
\item
  BRAILLARD, P. y DE SENARCLENS, P. ``EL IMPERIALISMO'', FCE, México, 1982.
\item
  BRUNSCHWIG, H. ``LE PARTAGE DE L'AFRIQUE NOIRE'', Flammarion, París, 1971.
\item
  CARDOZO, F.H. y FALETTO, E. "DEPENDENCIA Y DESARROLLO EN AMERICA LATINA,
  Siglo XXI, México, 1969.
\item
  CARTWRIGHT, Dorwin y ZANDER, Alvin ``GROUP DYNAMICS: RESEARCH AND THEORY'',
  Ed. Harper and Row, 1962.
\item
  CASSIRER, Ernest ``EL MITO DEL ESTADO'', FCE, México, 1968. CESAREO, Vincenzo et al.
  ``LA CULTURA DELL'ITALIA CONTEMPORANEA. TRASFORMAZIONE DEI MODELLI DI
  COMPORTAMENTO E IDENTITA SOCIALE'', Ed. Fondazione Giovanni Agnelli, Torino, 1990.
\item
  CHATELET, F., DUHAMEL, O. y PISIER, E. ``DICTIONNAIRE DES OEUVRES POLITIQUES'',
  P.U.F., París, 1989.- CHEVALIER, J.J. ``LOS GRANDES TEXTOS POLITICOS - DESDE MAQUIAVELO A
  NUESTROS DIAS'', Aguilar, Madrid, 1979.
\item
  COPLIN, W.D. "INTRODUCTION TO INTERNATIONAL POLITICS. A THEORETICAL
  OVERVIEW, Chicago ,1971.
\item
  CROZIER, Michel ``LE PHENOMENE BUREAUCRATIQUE'', Seuil, París, 1964.
\item
  DENQUIN, Jean-Marie ``SCIENCE POLITIQUE'', P.U.F., París, 1991.
\item
  DEUTSCH, K.W. ``NATIONALISM AND SOCIAL COMMUNICATION. AN INQUIRY INTO
  THE FOUNDATIONS OF NATIONALITY'', M.I.T. Press, Mass., 1953.
\item
  DEUTSCH, K. et al.~``POLITICAL COMMUNITY AND THE NORTH ATLANTIC AREA.
  INTERNATIONAL ORGANIZATION IN THE LIGHT OF HISTORICAL EXPERIENCE'',
  Princeton, 1957.
\item
  DEUTSCH, Karl ``POLITICA Y GOBIERNO'', FCE, México, 1976.
\item
  DEUTSCH, Karl ``LOS NERVIOS DEL GOBIERNO'', FCE, México, 1985.
\item
  DEUTSCH, Morton y KRAUSS, Robert ``THEORIES IN SOCIAL PSYCHOLOGY'', Basic Books,
  Inc., 1965.
\item
  DIAMANT, A. ``THE NATURE OF POLITICAL DEVELOPMENT'' en Finkle y Gable
  ``POLITICAL DEVELOPMENT AND SOCIAL CHANGE'', John Wiley, New York, 1966.
\item
  DJILAS, Milovan ``LA NUEVA CLASE'', Sudamericana, Bs.As., 1965.
\item
  DRAPER, T. ``ABUSE OF POWER'', Secker and Warburg, London, 1966.
\item
  DURKHEIM, E. "DE LA DIVISION DEL TRABAJO SOCIAL, Schapire, Bs.As, 1967.
\item
  ------------ ``EL SUICIDIO'', Schapire, Bs.As., 1965.
\item
  ------------ ``LAS FORMAS ELEMENTALES DE LA VIDA RELIGIOSA'', Schapire, Bs.As., 1968.
\item
  EASTON ,D. y DENNIS, J. ``CHILDREN IN THE POLITICAL SYSTEM'', New York, 1969.
\item
  EASTON, David "ESQUEMA PARA EL ANALISIS POLITICO, Amorrortu, Bs.As., 1969.
\item
  ECKSTEIN, H. y APTER, D. ``COMPARATIVE POLITICS. A READER'', New York, 1963.
\item
  ECKSTEIN, H. ``DIVISION AND COHESION IN DEMOCRACY - A STUDY OF NORWAY'',
  Princeton University Press, 1966.
\item
  EISENSTADT, S.N. ``MODERNIZACION, MOVIMIENTOS DE PROTESTA Y CAMBIO
  SOCIAL'', Amorrortu, Bs.As., 1969.
\item
  EMMANUEL, Arghiri "L'ECHANGE INEGAL. ESSAI SUR LES ANTAGONISMES DANS LES
  RAPPORTS ECONOMIQUES INTERNATIONAUX, Maspero, París, 1969.
\item
  FEJTÖ, François ``LA SOCIAL-DEMOCRATIE QUAND MEME'', Ed. Robert Laffont, París, 1980. - FESTINGER, Leo ``A THEORY OF COGNITIVE DISSONANCE'', Row, Peterson and Co., 1957.
\item
  FRANK, André Gunder ``CAPITALISMO Y SUBDESARROLLO EN AMERICA LATINA'', Siglo
  XXI, México, 1970.
\item
  FRENCH, John R.P. Jr.~``A FORMAL THEORY OF SOCIAL POWER'' en Cartwright y Zander:
  ``GROUPS DINAMICS: RESEARCH AND THEORY'', Ed. Harper and Row, 1962.
\item
  FREUD, Sigmund ``OBRAS COMPLETAS'' , Tomos II y III, Ed. Biblioteca Nueva, Madrid, 1973.
\item
  FRIEDRICH, Carl ``EL HOMBRE Y EL GOBIERNO: UNA TEORIA EMPIRICA DE LA
  POLITICA'', Tecnos, Madrid, 1968.
\item
  FRIEDRICH, C. y BRZEZINSKI, Z. ``DICTADURA TOTALITARIA Y AUTOCRACIA'', Libera,
  Bs.As., 1975.
\item
  FULBRIGHT, J.W. ``THE ARROGANCE OF POWER'', Vintage Books, New York, 1966.
\item
  FURTADO, Celso ``DESARROLLO Y SUBDESARROLLO'', Eudeba, Bs.As., 1965.
\item
  GALLAGER, John y ROBINSON, Ronald ``AFRICA AND THE VICTORIANS. THE OFFICIAL
  MIND OF IMPERIALISM'', Ed. Macmillan, London, 1961.
\item
  GARCIA PELAYO, Manuel ``MITOS Y SIMBOLOS POLITICOS'', Taurus, Madrid, 1964.
\item
  GENIAGE, Jean ``L'EXPANSION COLONIALE DE LA FRANCE SOUS LA IIIe REPUBLIQUE
  (1871-1914)'', Payot, París, 1968.
\item
  GERMANI, Gino ``POLITICA Y SOCIEDAD EN UNA EPOCA DE TRANSICION'', Paidós,
  Bs.As., 1965.
\item
  GOLEMBIEWSKI, Robert ``BEHAVIOR AND ORGANIZATION: ORGANIZATION AND
  METHODS AND THE SMALL GROUP'', Rand McNally and Co., 1962.
\item
  GORI, U., BRUSCHI, A., ATTINA, F. ``RELAZIONI INTERNAZIONALI. METODI E TECNICHE
  DI ANALISI'', Milán, 1974.
\item
  GRAMSCI, Antonio ``NOTAS SOBRE MAQUIAVELO, LA POLITICA Y EL ESTADO'', Juan
  Pablos, México, 1975.
\item
  GURVITCH, Georges ``TRATADO DE SOCIOLOGIA'', Kapeluz, BS.As., 1963.
\item
  ----------------- ``LES CADRES SOCIAUX DE LA CONNAISSANCE'', PUF, París, 1966.
\item
  HABERMAS, Jürgen ``TEORIA Y PRAXIS'', Sur, Bs.As., 1967.
\item
  ---------------- "TEORIA E PRASSI NELLA SOCIETA TECNOLOGICA, Bari, 1969.
\item
  HARGROVE, E.C. ``PRESIDENTIAL LEADERSHIP - PERSONALITY AND POLITICAL
  STYLE'', New York/London, 1966. - HAURIOU, A. ``DERECHO CONSTITUCIONAL E INSTITUCIONES POLITICAS'', Ed. Ariel,
  Barcelona, 1971.
\item
  HEARNSHAW, F.J.C. ``HISTORIA DE LAS IDEAS POLITICAS'', Empresa Letras, Santiago de
  Chile, .
\item
  HILFERDING, Rudolf ``LE CAPITAL FINANCIER. ETUDE SUR LE DEVELOPMENT RECENT
  DU CAPITALISME'', Ed. du Minuit, París, 1970.
\item
  HOBSON, J.A. ``ESTUDIOS DEL IMPERIALISMO'', Alianza, Madrid, 1981.
\item
  HOFFMAN, Stanley ``GULLIVER EMPETRÉ. ESSAI SUR LA POLITIQUE ETRANGERE DES
  ETATS-UNIS'', Seuil, París, 1971.
\item
  HOLT, R.T. y TURNER E. ``THE METHODOLOGY OF COMPARATIVE RESEARCH'', New
  York, 1970.
\item
  HOMANS, George C. ``SOCIAL BEHAVIOR: ITS ELEMENTARY FORMS'', Harcourt, Brace and
  Word Inc., 1961.
\item
  HORKHEIMER, Max y ADORNO, Theodor ``DIALECTICA DEL ILUMINISMO'', Sur, Bs.As.,
\end{itemize}

\begin{enumerate}
\def\labelenumi{\arabic{enumi}.}
\setcounter{enumi}{1968}
\tightlist
\item
\end{enumerate}

\begin{itemize}
\tightlist
\item
  HOVLAND, Car I. et al.~``COMMUNICATION AND PERSUATION: PSYCOLOGICAL STUDIES
  OF OPINION CHANGE'', Yale University Press, 1953.
\item
  HULL, Clark L. ``A BEHAVIOR SYSTEM'', Yale University Press, 1952.
\item
  HUNTINGTON, S.P. y MOORE, C.H. ``AUTHORITARIAN POLITICS IN MODERN SOCIETY'',
  New York, 1970.
\item
  HUNTINGTON, S.P. ``EL ORDEN POLITICO EN LAS SOCIEDADES EN CAMBIO'', Paidós,
  Bs.As., 1972.
\item
  JAGUARIBE H. et al.~``LA DEPENDENCIA POLITICO-ECONOMICA DE AMERICA LATINA'',
  Siglo XXI, México, 1971.
\item
  JAGUARIBE-FURTADO-FALETTO-DITELLA-ESPARTACO-SUNKEL- CARDOSO ``LA
  DOMINACION DE AMERICA LATINA'', Amorrortu, Bs.As., 1972.
\item
  JAGUARIBE, Helio ``SOCIEDAD, CAMBIO Y SISTEMA POLITICO'', Paidós, Bs.As., 1972.
\item
  ---------------- ``DESARROLLO POLITICO - SENTIDO Y CONDICIONES'', Paidós, Bs.As., 1972.
\item
  ---------------- ``AMERICA LATINA - REFORMA O REVOLUCION'', Paidós, Bs.As., 1972.
\item
  ---------------- ``O NOVO CENARIO INTERNACIONAL'', Ed. Guanabara, Río de Janeiro, 1986.
\item
  JALEÉ, Pierre ``L'IMPERIALISME EN 1970'', Maspero, París, 1973.
\item
  JAMES. Emile ``HISTORIA DEL PENSAMIENTO ECONOMICO'', Aguilar, Madrid, 1974. - JOUVENEL, Bertrand de ``EL PODER'', Ed. Nacional, Madrid, 1974.
\item
  2a ed.~KAPLAN, M.A.~``SYSTEM AND PROCES IN INTERNATIONAL POLITICS'', New York,
\end{itemize}

\begin{enumerate}
\def\labelenumi{\arabic{enumi}.}
\setcounter{enumi}{1956}
\tightlist
\item
\end{enumerate}

\begin{itemize}
\tightlist
\item
  KEOHANE, R.O. y NYE, J.S. ``TRANSNATIONAL RELATIONS IN WORLD POLITICS'',
  Harvard University Press, 1972.
\item
  KNOLL, E. y McFADEN, J. ``AMERICAN MILITARISM - 1970'', The Viking Press, New York,
\end{itemize}

\begin{enumerate}
\def\labelenumi{\arabic{enumi}.}
\setcounter{enumi}{1968}
\tightlist
\item
\end{enumerate}

\begin{itemize}
\tightlist
\item
  KORNHAUSER, William ``ASPECTOS POLITICOS DE LA SOCIEDAD DE MASAS'', Amorrortu,
  Bs.As., 1969.
\item
  LAGROYE, Jacques ``SOCIOLOGIE POLITIQUE'', Presses de la F.N. des Sc. Po. \& Dalloz, París,
\end{itemize}

\begin{enumerate}
\def\labelenumi{\arabic{enumi}.}
\setcounter{enumi}{1990}
\tightlist
\item
\end{enumerate}

\begin{itemize}
\tightlist
\item
  LANE, R. ``POLITICAL IDEOLOGY'', New York, 1962.
\item
  LAPLANCHE, J. y PONTALIS, J.B. ``DICCIONARIO DE PSICOANALISIS'', Ed. Labor,
  Barcelona, 1974.
\item
  LASSWELL, Harold D. ``PSYCHOPATHOLOGY AND POLITICS'', Viking Press Inc., 1962.
\item
  LERNER, D. "THE PASSING OF TRADITIONAL SOCIETY. MODERNIZING THE MIDDLE
  EAST, New York, 1958.
\item
  LEWIN, Kurt ``FIELD THEORY IN SOCIAL SCIENCE'', Dorwin Cartwright (Harper and Bros.),
\end{itemize}

\begin{enumerate}
\def\labelenumi{\arabic{enumi}.}
\setcounter{enumi}{1950}
\tightlist
\item
\end{enumerate}

\begin{itemize}
\tightlist
\item
  LIJPHART, A. ``THE POLITICS OF ACCOMODATION. PLURALISM AND DEMOCRACY IN
  THE NETHERLANDS'', Berkeley, Los Angeles, 1968.
\item
  LIPSET, Seymour Martin ``EL HOMBRE POLITICO. LAS BASES SOCIALES DE LA POLITICA'',
  Eudeba ed., Bs.As., 1977.
\item
  LISKA, George ``IMPERIAL AMERICA. THE INTERNATIONAL POLITICS OF PRIMACY'',
  John Hopkins Press, Baltimore, 1967.
\item
  LOCKE, John ``ENSAYO SOBRE EL GOBIERNO CIVIL'', Aguilar, Madrid, 1981.
\item
  LOPEZ, Mario Justo ``INTRODUCCION A LOS ESTUDIOS POLITICOS'', Tomos I y II, Kapeluz,
  Bs.As., 1975.
\item
  LUKACS, György ``EL ASALTO A LA RAZON'', Grijalbo, México, 1976.
\item
  LUXEMBURG, Rosa ``LA ACUMULACION DEL CAPITAL'', Grijalbo, México, 1967.
\item
  MAIMONIDES, M. ``THE GUIDE OF THE PERPLEXED'', University of Chica-go Press, Chicago,
\end{itemize}

\begin{enumerate}
\def\labelenumi{\arabic{enumi}.}
\setcounter{enumi}{1962}
\item
  \begin{itemize}
  \tightlist
  \item
    MANNHEIM, Karl ``IDEOLOGIA Y UTOPIA'', Aguilar, Madrid, 1973.
  \end{itemize}
\end{enumerate}

\begin{itemize}
\tightlist
\item
  MAQUIAVELO, Nicolás ``EL PRINCIPE'', Alianza ed., Madrid, 1981.
\item
  ------------------- ``DISCURSOS SOBRE LA PRIMERA DECADA DE TITO LIVIO'', en ``Obras'',
  Vergara, Barcelona, 1965.
\item
  MARCH, James G. y SIMON, Herbert A. ``ORGANIZATIONS'', John Wiley and Sons, 1962.
\item
  MARX, Karl ``LA IDEOLOGIA ALEMANA'', Grijalbo, México, 1969.
\item
  MARX, Karl ``ELEMENTOS FUNDAMENTALES PARA LA CRITICA DE LA ECONO- MIA
  POLITICA'', Siglo XXI, Madrid, 1972.
\item
  ---------- ``TEORIAS SOBRE LA PLUSVALIA'', FCE, México, 1982.
\item
  MARCUSE, Herbert ``EL FIN DE LA UTOPIA'', Siglo XXI, México, 1968.
\item
  ---------------- ``EL HOMBRE UNIDIMENSIONAL'', Mortiz, México, 1970.
\item
  MEEHAN, E.J. ``PENSAMIENTO POLITICO CONTEMPORANEO'', Rev.~de Occidente, Madrid,
\end{itemize}

\begin{enumerate}
\def\labelenumi{\arabic{enumi}.}
\setcounter{enumi}{1972}
\tightlist
\item
\end{enumerate}

\begin{itemize}
\tightlist
\item
  MERTON, Robert K. ``TEORIA Y ESTRUCTURAS SOCIALES'', FCE, México, 1964.
\item
  MICHELS, Robert ``LOS PARTIDOS POLITICOS'', Amorrortu, Bs.As., 1969
\item
  MILBRATH, Lester W. ``POLITICAL PARTICIPATION'', Rand Mc Nally and Co., 1965.
\item
  MOONEY, Alfredo y ARNOLETTO, Eduardo ``CUESTIONES FUNDAMENTALES DE CIENCIA
  POLITICA'', Alveroni ed., Córdoba, 1993.
\item
  MORGENTHAU, Hans ``POLITICS AMONG NATIONS. THE STRUGGLE FOR POWER AND
  PEACE'', Knopf, New York, 1955.
\item
  MORLINO, L. Comp. ``GUIDE AGLI STUDI DI SCIENZE SOCIALI IN ITALIA - SCIENZA
  POLITICA'', Ed. Fond. G. Agnelli, Torino, 1989.
\item
  MORO, Tomás ``UTOPIA'', Bruguera, Barcelona, 1973.
\item
  ORGANSKI, A.F.K. ``THE STAGES OF POLITICAL DEVELOPMENT'', A. Knopf, New York,
\end{itemize}

\begin{enumerate}
\def\labelenumi{\arabic{enumi}.}
\setcounter{enumi}{1964}
\tightlist
\item
\end{enumerate}

\begin{itemize}
\tightlist
\item
  OSSOWSKI, Stanislaw ``ESTRUCTURA DE CLASE Y CONCIENCIA SOCIAL'', Península,
  Barcelona, 1971.
\item
  PARETO, Vilfredo ``FORMA Y EQUILIBRIO SOCIALES'', Rev.~de Occidente, Madrid, 1966.
\item
  PARETI, Luigi et al.~``HISTORIA DE LA HUMANIDAD - DESARROLLO CULTURAL Y
  CIENTIFICO'', Tomo II (UNESCO), Ed. Sudamericana, Bs.As., 1969.
\item
  PARTRIDGE, P.H. ``CONSENT AND CONSENSUS'', Londres, 1971.
\item
  PARSONS, Talcott ``EL SISTEMA SOCIAL'', Rev.~de Occidente, Madrid, 1976. - ---------------- ``ENSAYOS DE TEORIA SOCIOLOGICA'', Paidós, Bs.As., 1970.
\item
  ---------------- ``EL SISTEMA DE LAS SOCIEDADES MODERNAS'', Trillas México, 1974.
\item
  PITKIN, H. ``THE CONCEPT OF REPRESENTATION'', Berkeley, 1967.
\item
  PINTO, A. ``POLITICA Y DESARROLLO'', Ed. Universitaria, Santiago de Chile, 1972.
\item
  PLATON ``LA REPUBLICA'', UNAM, México, 1971.
\item
  ------ ``LAS LEYES'', Inst. de Est. Políticos, Madrid, 1960.
\item
  ------ ``EL POLITICO'', Inst. de Est. Políticos, Madrid, 1955.
\item
  PUTNAM, R.D. "THE BELIEF OF POLITICIANS: IDEOLOGY, CONFLICT AND
  DEMOCRACY IN BRITAIN AND ITALY, London, 1973.
\item
  PYE, L.W. ``COMMUNICATIONS AND POLITICAL DEVELOPMENT'', Princeton, 1963.
\item
  --------- ``POLITICS, PERSONALITY AND NATION-BUIDING. BURMS'S SEARCH FOR
  IDENTITY'', Yale University Press, New Haven, 1966.
\item
  --------- ``ASPECTS OF POLITICAL DEVELOPMENT'', Little Brown, Boston, 1966.
\item
  PYE, L.W. y VERBA, S. ``POLITICAL CULTURE AND POLITICAL DEVELOP-MENT'',
  Princeton University Press, 1969.
\item
  RIBEIRO, Darsy ``EL DILEMA DE AMERICA LATINA'', Siglo XXI, México, 1971.
\item
  RICHARDSON, Lewis ``ARMS AND INSECURITY'', Quadrangle Press, Chicago, 1960.
\item
  ROBINSON, R. ``THE NON-EUROPEAN FOUNDATIONS OF EUROPEAN IMPERIALISM:
  SKETCH FOR A THEORY OF COLLABORATION'', Longman, London, 1972.
\item
  ROUQUIÉ, Alain ``EXTREMO OCCIDENTE. INTRODUCCION A AMERICA LATINA'', Emecé,
  Bs.As., 1991.
\item
  ROSTOW, Walt ``LAS ETAPAS DEL CRECIMIENTO ECONOMICO'', FCE, México, 1961.
\item
  ROSTOW, W.W. ``POLITICS AND THE STAGES OF GROWTH'', Cambridge, 1971.
\item
  ROSENAU, J.N. ``THE SCIENTIFIC STUDY OF FOREIGN POLICY'', New York, 1971.
\item
  RUNCIMAN, W.G. ``SOCIOLOGY IN ITS PLACE'', Cambridge, 1970.
\item
  RUYER, Raymond ``L'UTOPIE ET LES UTOPIES'', PUF, París, 1950.
\item
  SABINE, G.H. ``HISTORIA DE LA TEORIA POLITICA'', FCE, México, 1984.
\item
  SANTOS, Theodoro dos ``LA NUOVA DIPENDENZA'', Milán, 1971.
\item
  SARTORI, G. ``SISTEMI RAPPRESENTATIVI'' en ``DEMOCRAZIA E DEFINIZIO- NI'', Bolonia,
\end{itemize}

\begin{enumerate}
\def\labelenumi{\arabic{enumi}.}
\setcounter{enumi}{1968}
\tightlist
\item
\end{enumerate}

\begin{itemize}
\tightlist
\item
  ----------- ``LA POLITICA - LOGICA Y METODO EN LAS CIENCIAS SOCIA- LES'', FCE,
  México, 1984. - SCHLESINGER, A.M. ``THE IMPERIAL PRESIDENCY'', Popular Library, New York, 1974.
\item
  SCHUMPETER, Joseph ``CAPITALISMO, SOCIALISMO Y DEMOCRACIA'', Agui- lar, México,
\end{itemize}

\begin{enumerate}
\def\labelenumi{\arabic{enumi}.}
\setcounter{enumi}{1960}
\tightlist
\item
\end{enumerate}

\begin{itemize}
\tightlist
\item
  ------------------ ``IMPERIALISMO Y CLASES SOCIALES'', Tecnos, Ma- drid, 1965.
\item
  SCHMITT, Carl ``LEGALIDAD Y LEGITIMIDAD'', Aguilar, Madrid, . SAINT-SIMON
  ``OEUVRES'', Anthropos, París, 1966.
\item
  SIMON, Herbert A. ``MODELS OF MAN: SOCIAL AND RATIONAL'', Wiley, New York, 1957.
\item
  SKINNER, B.F. ``SCIENCE AND HUMAN BEHAVIOR'', Free Press, New York, 1953.
\item
  SOREL, Jean ``REFLEXIONES SOBRE LA VIOLENCIA'', Alianza, Madrid, 1976.
\item
  SUNKEL, O. y PAZ, P. ``EL DESARROLLO LATINOAMERICANO Y LA TEORIA DEL
  DESARROLLO'', Siglo XXI, México, 1970.
\item
  SWEEZY, Paul ``ÉLITE DE PODER O CLASE DIRIGENTE?'', Jorge Alvarez, Bs.As., . TUCKER,
  Robert ``NATION OR EMPIRE? DEBATE OVER AMERICAN FOREIGN POLICY'', J. Hopkins
  Press, Baltimore, 1968.
\item
  TZU, Sun ``L'ART DE LA GUERRE'', Flammarion, París, 1972.
\item
  URBANI, G. ``LA POLITICA COMPARATA'', Bolonia, 1972.
\item
  VERBA, Sidney ``SMALL GROUPS AND POLITICAL BEHAVIOR'', Princeton University Press,
\end{itemize}

\begin{enumerate}
\def\labelenumi{\arabic{enumi}.}
\setcounter{enumi}{1960}
\tightlist
\item
\end{enumerate}

\begin{itemize}
\tightlist
\item
  VOEGELIN, Eric ``NUEVA CIENCIA DE LA POLITICA'', Rialp, Madrid, 1968.
\item
  VRANICKI, P., SUPEK, R. et al.~``EL SOCIALISMO YUGOESLAVO ACTUAL'', Grijalbo ed.,
  México, 1975.
\item
  WEBER, Max ``EL POLITICO Y EL CIENTIFICO'', Alianza, Madrid, 1967.
\item
  ---------- ``ECONOMIA Y SOCIEDAD'', FCE, México, 1964.
\item
  YARMOLINSKY, Adam ``THE MILITARY ESTABLISHMENT. ITS IMPACT ON AMERICAN
  SOCIETY'', Harper \& Row, New York, 1971.
\item
  ZEITLIN, Irving ``IDEOLOGIA Y TEORIA SOCIOLOGICA'', Amorrortu ed., Bs.As., 1973
\end{itemize}

  \bibliography{book.bib,packages.bib}

\end{document}
